\documentclass[12pt,landscape]{article}
\usepackage{multicol}
\usepackage{calc}
\usepackage{ifthen}
\usepackage[landscape]{geometry}
\usepackage{hyperref}

% To make this come out properly in landscape mode, do one of the following
% 1.
%  pdflatex latexsheet.tex
%
% 2.
%  latex latexsheet.tex
%  dvips -P pdf  -t landscape latexsheet.dvi
%  ps2pdf latexsheet.ps


% If you're reading this, be prepared for confusion.  Making this was
% a learning experience for me, and it shows.  Much of the placement
% was hacked in; if you make it better, let me know...


% 2008-04
% Changed page margin code to use the geometry package. Also added code for
% conditional page margins, depending on paper size. Thanks to Uwe Ziegenhagen
% for the suggestions.

% 2006-08
% Made changes based on suggestions from Gene Cooperman. <gene at ccs.neu.edu>


% To Do:
% \listoffigures \listoftables
% \setcounter{secnumdepth}{0}


% This sets page margins to .5 inch if using letter paper, and to 1cm
% if using A4 paper. (This probably isn't strictly necessary.)
% If using another size paper, use default 1cm margins.
\ifthenelse{\lengthtest { \paperwidth = 11in}}
	{ \geometry{top=.5in,left=.5in,right=.5in,bottom=.5in} }
	{\ifthenelse{ \lengthtest{ \paperwidth = 297mm}}
		{\geometry{top=1cm,left=1cm,right=1cm,bottom=1cm} }
		{\geometry{top=1cm,left=1cm,right=1cm,bottom=1cm} }
	}

% Turn off header and footer
\pagestyle{empty}
 

% Redefine section commands to use less space
\makeatletter
\renewcommand{\section}{\@startsection{section}{1}{0mm}%
                                {-1ex plus -.5ex minus -.2ex}%
                                {0.5ex plus .2ex}%x
                                {\normalfont\large\bfseries}}
\renewcommand{\subsection}{\@startsection{subsection}{2}{0mm}%
                                {-1explus -.5ex minus -.2ex}%
                                {0.5ex plus .2ex}%
                                {\normalfont\normalsize\bfseries}}
\renewcommand{\subsubsection}{\@startsection{subsubsection}{3}{0mm}%
                                {-1ex plus -.5ex minus -.2ex}%
                                {1ex plus .2ex}%
                                {\normalfont\small\bfseries}}
\makeatother

% Define BibTeX command
\def\BibTeX{{\rm B\kern-.05em{\sc i\kern-.025em b}\kern-.08em
    T\kern-.1667em\lower.7ex\hbox{E}\kern-.125emX}}

% Don't print section numbers
\setcounter{secnumdepth}{0}


\setlength{\parindent}{0pt}
\setlength{\parskip}{0pt plus 0.5ex}


% -----------------------------------------------------------------------

\begin{document}

\raggedright
\footnotesize
\begin{multicols}{3}


% multicol parameters
% These lengths are set only within the two main columns
%\setlength{\columnseprule}{0.25pt}
\setlength{\premulticols}{1pt}
\setlength{\postmulticols}{1pt}
\setlength{\multicolsep}{1pt}
\setlength{\columnsep}{2pt}

\begin{center}
     \Large{\textbf{Clojure Emacs \\ Reference Card}} \\
\end{center}.

\section{Clojure Buffer}
\begin{tabular}{p{6cm}p{1cm}} % {@{}ll@{}}
	Evalulate the form preceding point    & \verb!C-x C-e!  \\
	Evaluate the top level form under point & \verb!C-M-x! \\
	Evaluate the region & \verb!C-c C-r! \\
	Interrupt any pending evaluations & \verb!C-c C-b! \\
	Invoke macroexpand-1 on the form preceding point & \verb!C-c C-m! \\
	Invoke clojure.walk/macroexpand-all on the form preceding point & \verb!C-c M-m! \\
	Eval the ns form & \verb!C-c C-n! \\
	Switch the namespace of the repl buffer to the namespace of the current buffer & \verb!C-c M-n! \\
	Select the repl buffer & \verb!C-c C-z! \\
	Clear the entire REPL buffer, leaving only a prompt & \verb!C-c M-o! \\
	Load the current buffer & \verb!C-c C-k! \\
	Load a file & \verb!C-c C-l! \\
	Display doc string for the symbol at point & \verb!C-c C-d! \\
	Display the source for the symbol at point & \verb!C-c C-s! \\
	Display JavaDoc (in your default browser) for the symbol at point & \verb!C-c C-j! \\
	Jump to the definition of a var & \verb!M-.! \\
	Return to your pre-jump location & \verb!M-,! \\
	Complete the symbol at point & \verb!M-TAB! \\
	Slurp the next expression into this expression & \verb!C-)! \\
	Slurp the previous expression into this expression & \verb!C-(! \\
	Barf the current expression out to the right of its parent expression & \verb!C-\}! \\
	Barf the current expression out to the left of its parent expression & \verb!C-\{! \\
	
\end{tabular}


%For spacing reasons - remove later
\begin{center}
	\\
\end{center}.

\section{REPL Buffer}
\begin{tabular}{p{6cm}p{1cm}} % {@{}ll@{}}

  Evaluate the current input in Clojure if it is complete & \verb!RET! \\
  Close any unmatched parenthesis and then evaluate the current input in Clojure & \verb!C-RET! \\
  Open a new line and indent & \verb!C-j! \\
  Clear the entire REPL buffer, leaving only a prompt & \verb!C-c M-o! \\
  Remove the output of the previous evaluation from the REPL buffer & \verb!C-c C-o! \\
  Kill all text from the prompt to the current point & \verb!C-c C-u! \\
  Interrupt any pending evaluations & \verb!C-c C-b! \\
  Goto to previous input in history & \verb!C-up! \\
  Goto to next input in history & \verb!C-down! \\
  Search the previous item in history using the current input as search pattern & \verb!M-p! \\
  Search the next item in history using the current input as search pattern & \verb!M-n! \\
  Search {\bf forwards } through command history with regex & \verb!M-s\}! \\
  Search {\bf backwards} through command history with regex & \verb!M-r!\} \\
  Move between the current prompts in the REPL buffer & \verb!C-c C-n! \\
  Move between the previous prompts in the REPL buffer & \verb!C-c C-p! \\
  Complete symbol at point & \verb!TAB! \\
  Display doc string for the symbol at point & \verb!C-c C-d! \\
  Display JavaDoc (in your default browser) for the symbol at point & \verb!C-c C-j! \\
\end{tabular}




\section{Macroexpansion Buffer}
\begin{tabular}{p{6cm}p{1cm}} % {@{}ll@{}}
	
	Invoke macroexpand-1 on the form preceding point & \verb!C-c C-m! \\
	Invoke clojure.walk/macroexpand-all on the form preceding point & \verb!C-c M-m! \\
	The prior macroexpansion is performed again & \verb!g! \\
	Undo the last inplace expansion performed & \verb!C-/, C-x u! \\
	
\end{tabular}

\rule{0.3\linewidth}{0.25pt}
\scriptsize

\href{https://github.com/humphrej/clojure-emacs-refcard}{https://github.com/humphrej/clojure-emacs-refcard}


\end{multicols}
\end{document}
